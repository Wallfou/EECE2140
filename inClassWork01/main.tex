% EECE 2140 – Assignment #01 (Setup & Tools Check) Template
% Overleaf-ready LaTeX template for students
% -----------------------------------------------------------------------------
\documentclass[11pt]{article}

% --- Basic packages ---
\usepackage[margin=1in]{geometry}
\usepackage{lmodern}
\usepackage[T1]{fontenc}
\usepackage{microtype}
\usepackage{enumitem}
\usepackage{hyperref}
\usepackage{graphicx}
\usepackage{array}
\usepackage{booktabs}
\usepackage{multicol}
\usepackage{xcolor}
\usepackage{tcolorbox}
\usepackage{listings}
\usepackage{caption}
\usepackage{amsmath, amssymb}

% --- Hyperref setup ---
\hypersetup{
  colorlinks=true,
  urlcolor=blue,
  linkcolor=blue,
  citecolor=blue
}

% --- Listings (for code/terminal) ---
\lstdefinestyle{term}{
  basicstyle=\ttfamily\small,
  keywordstyle=\color{blue!70!black},
  stringstyle=\color{green!40!black},
  commentstyle=\color{black!50},
  numberstyle=\tiny\color{black!50},
  numbers=left,
  stepnumber=1,
  numbersep=8pt,
  showstringspaces=false,
  frame=single,
  framerule=0.3pt,
  breaklines=true,
  tabsize=2
}

% --- Check box style ---
\tcbset{colback=gray!5,colframe=black!50,arc=1mm,outer arc=1mm}
\newtcolorbox{checkbox}[1][]{
  title=\textbf{Checklist Item},#1
}

% --- Useful macros ---
\newcommand{\coursename}{EECE 2140: Computing Fundamentals for Engineers}
\newcommand{\semester}{Spring 2026}
\newcommand{\assignmentno}{01}
\newcommand{\instructor}{Dr.~Fatema Nafa}

% Fill these out before you begin
\newcommand{\studentname}{<Your Name Here>}
\newcommand{\nuid}{<Your NUID>}
\newcommand{\email}{<your.email@northeastern.edu>}

% --- Task environment ---
\newcounter{task}
\newenvironment{task}[1][]{%
  \refstepcounter{task}%
  \par\vspace{0.75em}
  {\large\bfseries Task~\thetask.}\ \\[-0.25em]
  \textit{#1}\par\vspace{0.25em}
}{\vspace{0.75em}}

% --- Document start ---
\begin{document}

% --- Logo ---
\begin{center}
  \includegraphics[width=0.58\textwidth]{logo.png}\\[1.85em]
\end{center}

\begin{center}
  {\Large \textbf{\coursename}}\\[2pt]
  {\large \semester}\\[8pt]
  {\Large \textbf{Lab Assignment \#\assignmentno\: Set up GitHub repo and VSCode/Makefile build environment on Linux}}\\[10pt]
  {\normalsize \textbf{Topic:} WSL2 + VS Code + GitHub + Overleaf}\\[10pt]
\end{center}


\begin{tabular}{@{}p{1.8in}p{4.6in}@{}}
\textbf{Student Name} & \studentname \\
\textbf{NUID} & \nuid \\
\textbf{Email} & \email \\
\textbf{Instructor} & \instructor \\
\textbf{Due Date} & <Enter Due Date> \\
\end{tabular}

\vspace{1em}
\noindent\textbf{Submission Checklist (must match syllabus requirements)}
\begin{itemize}[leftmargin=1.5em]
  \item Submit \textbf{one PDF} generated from this Overleaf report.
  \item Include \textbf{clear screenshots} for each task (text must be readable).
  \item Include your \textbf{GitHub profile link} and \textbf{Overleaf project link} in the report.
  \item File name: \texttt{Assignment\assignmentno\_LastName.pdf}
\end{itemize}

\vspace{0.5em}
\noindent\textbf{How to complete this assignment}
\begin{enumerate}[leftmargin=1.5em]
  \item Complete each task below in order.
  \item For each task, include the required evidence (screenshots, links, and short notes).
  \item When you paste a link, ensure it opens correctly (no private/local-only links unless instructed).
\end{enumerate}

\vspace{0.75em}
\hrule
\vspace{0.75em}

% -------------------------
% SECTION A: WSL2 CHECK
% -------------------------
\section*{Section A: WSL2 Installation Verification}

\begin{task}[Verify that WSL2 is installed and Ubuntu runs]
\begin{checkbox}
\textbf{Goal:} Confirm that WSL is installed, WSL2 is enabled, and your Linux distro launches.\\
\textbf{What to do (Windows PowerShell):}
\begin{itemize}[leftmargin=1.5em]
  \item Open \textbf{PowerShell} and run:
\end{itemize}
\end{checkbox}

\begin{lstlisting}[style=term,caption={WSL checks (PowerShell)}]
wsl --status
wsl -l -v
\end{lstlisting}

\noindent\textbf{Required evidence:}
\begin{itemize}[leftmargin=1.5em]
  \item Screenshot showing output of \texttt{wsl --status}
  \item Screenshot showing output of \texttt{wsl -l -v} (Version should be \texttt{2})
  \item Screenshot showing Ubuntu terminal open (your username visible)
\end{itemize}

\noindent\textbf{Notes (1--3 sentences):} \\
\textit{Write what you observe (e.g., distro name, default version, any issues).}
\end{task}

\begin{task}[Install build tools inside Ubuntu and verify versions]
\begin{checkbox}
\textbf{Goal:} Install the compiler and tools we will use all semester.\\
\textbf{What to do (Ubuntu terminal):} Run the commands below.
\end{checkbox}

\begin{lstlisting}[style=term,caption={Install tools (Ubuntu)}]
sudo apt update
sudo apt install -y build-essential cmake git
g++ --version
cmake --version
git --version
\end{lstlisting}

\noindent\textbf{Required evidence:}
\begin{itemize}[leftmargin=1.5em]
  \item Screenshot showing successful install command(s)
  \item Screenshot showing \texttt{g++}, \texttt{cmake}, and \texttt{git} version outputs
\end{itemize}

\noindent\textbf{Notes (1--3 sentences):}
\end{task}

% -------------------------
% SECTION B: VS CODE + WSL
% -------------------------
\section*{Section B: Visual Studio Code + WSL Integration}

\begin{task}[Install VS Code and connect it to WSL]
\begin{checkbox}
\textbf{Goal:} Use VS Code as your editor while compiling in WSL.\\
\textbf{What to do:}
\begin{itemize}[leftmargin=1.5em]
  \item Install VS Code on Windows.
  \item In VS Code, install the \textbf{WSL extension} (Microsoft).
  \item From Ubuntu terminal, open a folder in VS Code using \texttt{code .}
\end{itemize}
\end{checkbox}

\begin{lstlisting}[style=term,caption={Open VS Code from Ubuntu (WSL)}]
cd ~
mkdir -p eece2140
cd eece2140
code .
\end{lstlisting}

\noindent\textbf{Required evidence:}
\begin{itemize}[leftmargin=1.5em]
  \item Screenshot of VS Code showing bottom-left text \texttt{WSL: Ubuntu} (or similar)
  \item Screenshot of VS Code integrated terminal showing a Linux prompt (e.g., \texttt{username@...:~/eece2140\$})
\end{itemize}

\noindent\textbf{Notes (1--3 sentences):}
\end{task}

\begin{task}[Create and run a C++ ``Hello World'' from VS Code in WSL]
\begin{checkbox}
\textbf{Goal:} Confirm you can compile and run a C++ program using Linux tools.\\
\textbf{What to do:}
\begin{itemize}[leftmargin=1.5em]
  \item Create \texttt{main.cpp} in the \texttt{eece2140} folder.
  \item Compile and run using the terminal commands below.
\end{itemize}
\end{checkbox}

\begin{lstlisting}[style=term,caption={Compile and run (WSL terminal inside VS Code)}]
g++ -std=c++17 -Wall -Wextra main.cpp -o main
./main
\end{lstlisting}

\noindent\textbf{Required evidence:}
\begin{itemize}[leftmargin=1.5em]
  \item Screenshot of \texttt{main.cpp} open in VS Code
  \item Screenshot of terminal output showing successful run (your program output)
\end{itemize}

\noindent\textbf{Paste your C++ code here:}
\begin{lstlisting}[style=term,caption={main.cpp}]
#include <iostream>

int main() {
    std::cout << "Hello from EECE 2140 on WSL2!\n";
    return 0;
}
\end{lstlisting}

\noindent\textbf{Notes (1--3 sentences):}
\end{task}

% -------------------------
% SECTION C: GITHUB SETUP
% -------------------------
\section*{Section C: GitHub Account and First Repository}

\begin{task}[Create a GitHub account and set up your profile]
\begin{checkbox}
\textbf{Goal:} Create a GitHub account and confirm your profile is accessible.\\
\textbf{What to do:}
\begin{itemize}[leftmargin=1.5em]
  \item Create (or confirm) your GitHub account.
  \item Add a profile picture and your name.
\end{itemize}
\end{checkbox}

\noindent\textbf{Required evidence:}
\begin{itemize}[leftmargin=1.5em]
  \item Screenshot of your GitHub profile page
  \item Paste your GitHub profile link below
\end{itemize}

\noindent\textbf{GitHub Profile Link:} \\
\url{<Paste your GitHub profile URL here>}
\end{task}

\begin{task}[Create a repository and push your Hello World project]
\begin{checkbox}
\textbf{Goal:} Create a repo and push code from WSL using git.\\
\textbf{What to do (WSL terminal):} Use the command sequence below (edit placeholders).
\end{checkbox}

\begin{lstlisting}[style=term,caption={Initialize git repo and push (edit placeholders)}]
cd ~/eece2140
git init
git add main.cpp
git commit -m "Add Hello World"

# Replace with your repo URL from GitHub (HTTPS or SSH)
git branch -M main
git remote add origin <YOUR_REPO_URL_HERE>
git push -u origin main
\end{lstlisting}

\noindent\textbf{Required evidence:}
\begin{itemize}[leftmargin=1.5em]
  \item Screenshot of terminal showing successful \texttt{git push}
  \item Screenshot of your GitHub repo showing \texttt{main.cpp} uploaded
\end{itemize}

\noindent\textbf{GitHub Repository Link:} \\
\url{<Paste your repository URL here>}

\noindent\textbf{Notes (1--3 sentences):}
\end{task}

% -------------------------
% SECTION D: OVERLEAF SETUP
% -------------------------
\section*{Section D: Overleaf Setup and Report Submission}

\begin{task}[Create an Overleaf account and compile this report]
\begin{checkbox}
\textbf{Goal:} Confirm you can use Overleaf and compile a PDF.\\
\textbf{What to do:}
\begin{itemize}[leftmargin=1.5em]
  \item Create (or confirm) an Overleaf account.
  \item Upload this \texttt{.tex} file to a new Overleaf project.
  \item Click \textbf{Recompile} and download the PDF.
\end{itemize}
\end{checkbox}

\noindent\textbf{Required evidence:}
\begin{itemize}[leftmargin=1.5em]
  \item Screenshot of your Overleaf project (left file tree visible)
  \item Screenshot of successfully compiled PDF preview
\end{itemize}

\noindent\textbf{Overleaf Project Link (if shareable):} \\
\url{<Paste Overleaf project link here (optional if private)>}

\noindent\textbf{Notes (1--3 sentences):}
\end{task}

% -------------------------
% Appendix: Screenshots
% -------------------------
\clearpage
\appendix
\section*{Appendix: Screenshots}

\noindent Insert your screenshots here. Ensure they are readable and clearly labeled.

\vspace{0.5em}
\subsection*{A1: WSL status and distro version}
\includegraphics[width=\linewidth]{figs/wsl_status.png}

\vspace{0.5em}
\subsection*{A2: Tool versions in Ubuntu}
\includegraphics[width=\linewidth]{figs/tool_versions.png}

\vspace{0.5em}
\subsection*{B1: VS Code showing WSL: Ubuntu}
\includegraphics[width=\linewidth]{figs/vscode_wsl.png}

\vspace{0.5em}
\subsection*{B2: Hello World execution}
\includegraphics[width=\linewidth]{figs/hello_run.png}

\vspace{0.5em}
\subsection*{C1: GitHub profile and repo}
\includegraphics[width=\linewidth]{figs/github_repo.png}

\vspace{0.5em}
\subsection*{D1: Overleaf project + compiled PDF}
\includegraphics[width=\linewidth]{figs/overleaf_compile.png}

\vspace{1em}
\noindent\textit{Reminder:} Your screenshots must be clear, readable, and directly support each checklist item.

\end{document}
